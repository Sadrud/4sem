\documentclass{article}
\usepackage[utf8]{inputenc}
\usepackage[russian]{babel}
\usepackage{listings}
\usepackage{xcolor}
\usepackage{hyperref}

\definecolor{codegreen}{rgb}{0,0.6,0}
\definecolor{codegray}{rgb}{0.5,0.5,0.5}
\definecolor{codepurple}{rgb}{0.58,0,0.82}

\lstdefinestyle{mystyle}{
    backgroundcolor=\color{white},   
    commentstyle=\color{codegreen},
    keywordstyle=\color{magenta},
    numberstyle=\tiny\color{codegray},
    stringstyle=\color{codepurple},
    basicstyle=\ttfamily\footnotesize,
    breakatwhitespace=false,         
    breaklines=true,                 
    captionpos=b,                    
    keepspaces=true,                 
    numbers=left,                    
    numbersep=5pt,                  
    showspaces=false,                
    showstringspaces=false,
    showtabs=false,                  
    tabsize=2
}

\lstset{style=mystyle}

\begin{document}

\title{Система вычисления выпуклой оболочки множества точек}
\author{}
\date{}
\maketitle

\section{Описание проекта}
Репозиторий содержит реализацию алгоритма для поиска выпуклой оболочки множества точек. Система реализована по клиент-серверной архитектуре с тестами для проверки:

\begin{itemize}
    \item Корректности работы алгоритма выпуклой оболочки
    \item Передачи данных между клиентом и сервером
    \item Обработки граничных случаев
\end{itemize}

\section{Технологический стек}
\begin{tabular}{|l|l|}
\hline
\textbf{Компонент} & \textbf{Технология} \\
\hline
Язык программирования & C++ \\
Стандартная библиотека & STL \\
Система сборки & CMake, Make \\
Контроль версий & Git \\
\hline
\end{tabular}

\section{Инструкция по запуску}
\subsection{Требования}
\begin{itemize}
    \item Компилятор C++ (g++ или clang)
    \item CMake версии 3.10 или выше
    \item Git для управления версиями
\end{itemize}

\subsection{Сборка проекта}
Запустите эти команды по очереди
\begin{lstlisting}[language=bash]

git clone https://github.com/Sadrud/4sem.git
cd 4sem

cd build
cmake ..
make
\end{lstlisting}

\subsection{Запуск системы}
Для работы системы требуется запустить два компонента:

В двух терминалах запустите эти две команды
\begin{lstlisting}[language=bash]

./Server/GeometryServer


./Client/GeometryClient
\end{lstlisting}

\section{Архитектура системы}
Система состоит из следующих компонентов:

\begin{itemize}
    \item \textbf{Серверная часть} - реализует алгоритм вычисления выпуклой оболочки
    \item \textbf{Клиентская часть} - отправляет данные на сервер и получает результаты
    \item \textbf{Тесты} - проверяют корректность работы системы
\end{itemize}

\end{document}